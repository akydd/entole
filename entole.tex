\documentclass{article}
\usepackage{textgreek}
\usepackage{xspace}
\usepackage{booktabs}
\usepackage{tabularx}
\usepackage{longtable}
\usepackage[backend=biber,sortcites]{biblatex}

\newcommand{\entole}{\textepsilon\textnu\texttau\textomikron\textlambda\textepsilon\xspace}

\title{Applying machine learning to New Testament Greek}
\author{Alan Kydd}
\addbibresource{entole.bib}

\begin{document}

\maketitle

\begin{abstract}
The Koine Greek word \entole is often translated to the English
word ``commandment'' in the King James bible.  The word ``commandment'' itself
has various meanings depending on the context, as there are various tyoes of
``commandments'' in the Bible.  Sometimes the type
is obvious, such as when the ``commandment'' being referred to is explicity
stated.  Other times, the type must be inferred from other contextual
data.

We use use TensorFlow to train a deep neural
network to categorize the various translations of \entole,
and it's various forms, when the type of ``commandment'' is explicit. 
Then, for those instances when the type is not explicitly stated,  we use the network 
to determine the type.
\end{abstract}

\section{Introduction}
The word \entole, and it's variants, appear 67 times in the Greek New 
Testament, spanning 16 separate books.  \entole is often, but not always, translated as
``commandment'' in the King James Bible.  The type of commandment being referred to varies from text to 
text.

At times the meaning of ``commandment'' is clear from the context in which
the word appears.  For example, Todo.  However, there are times when the precise
meaning is not clear.  For example, Todo.  

\section{Methods}
36 of the total 67 occurences of \entole and its variants can each be explicitly linked to a specific
commandment or group of commandments.
We classify each of these 36 occurences according to the type of commandment, or commandments,
to which each occurence belongs: ``Torah'', ``Decalogue'' (itself a part
of the Torah), and ``Other''.  The ``Other'' type includes man-made laws
and instructions from one person to another, among others.

\begin{table}
    \footnotesize
    \renewcommand\arraystretch{1.2}
    \begin{tabularx}{\textwidth}{@{}llX@{}}
    \toprule
    Verse & Type & Reason \\
    \midrule
    Matthew 5:19 & Torah & Jesus uses the commandment as belonging to the Law \\
    Matthew 15:3 & Torah & Refers to Ex. 20:12, Ex. 21:17, Lev. 20:9 \\
    Matthew 19:17 & Torah & Five Decalogue commandments mentioned, plus Lev. 19:18 \\
    Matthew 22:38 & Torah & Lev. 19:18, Deut. 6:5\\
    Matthew 22:40 & Torah & Lev. 19:18, Deut. 6:5\\
    Mark 7:7 & Other & phrase is ``commandments of men'' \\
    Mark 7:8 & Torah & Verse 10 refers to Ex. 20:12, Ex. 21:17, Lev. 20:9\\
    Mark 7:9 & Torah & Verse 10 refers to Ex. 20:12, Ex. 21:17, Lev. 20:9\\
    Mark 10:5 & Torah & Refers to divorce law, Deut. 24:1 for example\\
    Mark 12:31 & Torah & In His response, Jesus refers to Lev. 19:18, Deut. 6:5\\
    Luke 15:29 & Other & Refers to a command given from father to son\\
    Luke 18:20 & Decalogue & Jesus refers to part of the Decalogue\\
    Luke 23:56 & Decalogue & Refers to Sabbath keeping\\
    John 10:18 & Other & Jesus refers to a commandment given to Him from God the Father\\
    John 11:57 & Other & Refers to command created by priests and Pharisees\\
    John 12:49 & Other & Jesus refers to a commandment given to Him from God the Father\\
    John 13:30 & Other & Jesus refers to a commandment given to Him from God the Father\\
    John 13:34 & Other & Jesus states that this is new commandment given by Him\\
    John 15:12 & Other & Jesus states that this is His commandment\\
    Acts 17:15 & Other & Command from Silas and Timothy\\
    Romans 7:8 & Decalogue & Refers to coveting, a member of the Decalogue\\
    Romans 13:9 & Decalogue & Word is used in reference to parts of the Decalogue.  ``Love thy neighbour'' is not called a commandment here.\\
    1 Cor. 14:37 & Other & Paul refers to his own writings, in this context, as the commands\\
    Eph. 6:2 & Decalogue & Honor your father and mother\\
    Col. 4:10 & Other & These are instructions to Mark when he visits\\
    1 Tim. 6:14 & Other & Refers to the author's instructions to Timothy, given in previous verses\\
    Titus 1:14 & Other & Usage is ``commandments of men''\\
    Heb. 7:5 & Torah & Refers to tithe\\
    Heb. 7:16 & Torah & Refers to laws defining the Levitical priesthood\\
    Heb. 7:18 & Torah & Refers to laws defining the Levitical priesthood\\
    Heb. 9:19 & Torah & May refer to Ex. 24:8.  Certainly not to the Decalogue\\
    1 John 3:23 & Other & The command here is to believe in the nme of Jesus, and to love one another\\
    1 John 3:23 & Other & The command here is to believe in the nme of Jesus, and to love one another\\
    1 John 3:24 & Other & Plural, referring to the above, but possibly includes more\\
    1 John 4:21 & Other & The command here is that anyone who loves God must also love their brother and sister\\
    2 John 1:5 & Other & The command here is to love one another\\
    \bottomrule
\end{tabularx}
    \caption{Classification of \entole}
    \label{training}
\end{table}

We use the set of classified data as the basis for the data set used to train a neural network to classify the remaining 34
occurences of \entole and its variants.  
We use a multilayer perceptron, implemented by TensorFlow's \texttt{tf.estimator.DNNClassifier}.

The ouput layer consists of three neurons, one for each type.

We use the following properties of each occurence for the features, or inputs, for the input layer:
Location of the occurence: if two occurences of \entole appear in the same verse, chapter, or book,
then perhaps the two occurences belong to the same type.  Traditional author of the text: who was
the author, and what was his/her cultural or religious background?  Given author of the text: similar
to the traditional author.  Details about dialogue, if any: was the word used in conversation?  If so,
who were the parties involved, and what was their cultural or religious background?  Intended audience:
who was the intended audience of the book in which the occurence appears?

\begin{table}
    \centering
    \renewcommand\arraystretch{1.3}
    \begin{tabular}{@{}lll@{}}
    \toprule
        Book & Traditional author & Traditional author's background \\
    \midrule
        Matthew & Matthew & Jewish Christian \\
        Mark & Mark & Jewish Christian \\
        Luke & Luke & Gentile Christian \\
        John & John & Jewish Christian \\
        Acts & Luke & Gentile Christian \\
        Romans & Paul & Jewish Christian \\
        1 Corintians & Paul & Jewish Christian \\
        Ephesians & Paul & Jewish Christian \\
        Collosians & Paul & Jewish Christian \\
        1 Timothy & Paul & Jewish Christian \\
        Titus & Paul & Jewish Christian \\
        Hebrews & Paul & Jewish Christian \\
        2 Peter & Peter & Jewish Christian \\
        1 John & John & Jewish Christian \\
        2 John & John & Jewish Christian \\
        Revelations & John & Jewish Christian \\
    \bottomrule
    \end{tabular}
    \caption{Traditional author and background}
    \label{traditionalauthor}
\end{table}

\begin{table}
    \centering
    \renewcommand\arraystretch{1.3}
    \begin{tabular}{@{}lll@{}}
    \toprule
        Book & Inferred author & Inferred author's background\\
    \midrule
        Matthew & Anon-Matthew & Jewish Christian~\cite{Duling2010},~\cite{Saldarini2003} \\
        Mark & Anon-Mark & Christian~\cite{Schroter2010} \\
        Luke & Anon-Luke & Gentile Christian \\
        John & Anon-John & Jewish Christian \\
        Acts & Anon-Luke & Gentile Christian \\
        Romans & Paul & Jewish Christian \\
        1 Corintians & Paul & Jewish Christian \\
        Ephesians & Pseudo-Paul-1 & Jewish Christian \\
        Colossians & Pseudo-Paul-2~\cite{Hooker2003} & Jewish Christian \\
        1 Timothy & Pseudo-Paul-3 & Jewish Christian \\
        Titus & Pseudo-Paul-3 & Jewish Christian \\
        Hebrews & Anon-Paul & Jewish Christian \\
        2 Peter & Pseudo-Peter & Jewish Christian \\
        1 John & Anon-John-2 & Jewish Christian \\
        2 John & Anon-John-2 & Jewish Christian \\
        Revelations & John of Patmos & Jewish Christian~\cite{Stuckenbruck2003} \\
    \bottomrule
    \end{tabular}
    \caption{Inferred author and background}
    \label{author}
\end{table}

\begin{table}
    \centering
    \renewcommand\arraystretch{1.3}
    \begin{tabular}{@{}ll@{}}
    \toprule
        Book & Audience \\
    \midrule
        Matthew & Jewish Christian~\cite{Saldarini2003} \\
        Mark & Gentile Christian~\cite{Schroter2010} \\
        Luke & Gentile \& Jewish Christian~\cite{Thompson2010} \\
        John & Jewish Christian \\
        Acts & Gentile Christian \\
        Romans & Jewish Christian \\
        1 Corintians & Gentile Christian \\
        Ephesians & Gentile Christian~\cite{Marshall2003} \\
        Colossians & Gentile Christian~\cite{Hooker2003} \\
        1 Timothy & Jewish Christian \\
        Titus & Gentile Christian~\cite{Perkins2003} \\
        Hebrews & Jewish Christian \\
        2 Peter & Jewish Christian \\
        1 John & Jewish Christian \\
        2 John & Jewish Christian \\
        Revelations & Jewish Christian \\
    \bottomrule
    \end{tabular}
    \caption{Audience}
    \label{audience}
\end{table}

\begin{table}
    \centering
    \renewcommand\arraystretch{1.3}
    \begin{tabular}{@{}lrrll@{}}
    \toprule
        Book & Chapter & Verse & Speaker & Listener \\
    \midrule
        Matthew & 5 & 19 & Jesus & crowds \\
         & 15 & 3 & Jesus & scribes and Pharisees \\
         & 19 & 17 & Jesus & rich young man \\
         & 22 & 36 & lawyer & Jesus \\
         &  & 38 & Jesus & lawyer \\
         &  & 40 & Jesus & lawyer \\
        Mark & 7 & 7 & Jesus & scribes and Pharisees \\
         &  & 8 & Jesus & scribes and Pharisees \\
         &  & 9 & Jesus & scribes and Pharisees \\
         & 10 & 5 & Jesus & Pharisees \\
         & & 19 & Jesus & rich young man \\
         & 12 & 28 & scribe & Jesus \\
         & & 31 & Jesus & scribe \\
        Luke & 15 & 29& Jesus & scribes and Pharisees \\
        & 18 & 20 & Jesus & rich young man \\
        John & 10 & 18 & Jesus & Pharisees \\
         & 12 & 49 & Jesus & crowds \\
         & 13 & 30 & Jesus & crowds \\
         & & 34 & Jesus & disciples \\
         & 14 & 15 & Jesus & disciples \\
         & & 21 & Jesus & disciples \\
         & 15 & 10 & Jesus & disciples \\
         & & 12 & Jesus & disciples \\
    \bottomrule
    \end{tabular}
    \caption{Dialogue}
    \label{dialogue}
\end{table}

We use a single hidden layer.  There are multiple methods of determining how many neurons
to include in this layer.  In this paper we use two rules of thumb provided in~\cite{Heaton2008}.

\begin{eqnarray}
    N_h & = & \frac{2 * N_i}{3} + N_o\\
    N_h & = & \frac{N_i + N_o}{2}
\end{eqnarray}
where $N_h$ is the number of neurons in the hidden layer, $N_i$ is the number of neurons in the input
layer, and $N_o$ is the number of neurons in the output layer.

\section{Results}
Here we can see that\ldots

\section{Discussion}
Test

\printbibliography
\end{document}
