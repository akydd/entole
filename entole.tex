\documentclass{article}
\usepackage{textgreek}
\usepackage{xspace}
\usepackage{booktabs}
\usepackage{tabularx}
\usepackage{longtable}
\usepackage[backend=biber,sortcites]{biblatex}
\usepackage{tablefootnote}

\newcommand{\entole}{\textepsilon\textnu\texttau\textomikron\textlambda\textepsilon\xspace}

\title{Applying machine learning to New Testament Greek}
\author{Alan Kydd}
\addbibresource{entole.bib}

\begin{document}

\maketitle

\begin{abstract}
The Koine Greek word \entole is often translated to the English
word ``commandment'' in the King James bible.  The word ``commandment'' itself
has various meanings depending on the context, as there are different types of
``commandments'' in the Bible.  Sometimes the type
is obvious, such as when the ``commandment'' being referred to is explicity
stated.  Other times, the type must be inferred from other contextual
data.

We use use TensorFlow to train a deep neural
network to categorize the various translations of \entole,
and it's various forms, when the type of ``commandment'' is explicit. 
Then, for those instances when the type is not explicitly stated,  we use the network 
to determine the type.
\end{abstract}

\section{Introduction}
The word \entole, and it's variants, appear 67 times in the Greek New 
Testament, spanning 16 separate books.  \entole is often, but not always,
translated as ``commandment'' in the King James Bible.  The type of
commandment being  referred to varies from verse to verse.

At times the ``commandment'' is provided in the verse in which
the word appears.  For example, Luke 23:56 refers to Sabbath keeping.
Other times, the ``commandment'' is provided in the surrounding verses,
as in Mark 7:8--9 when paired with Mark 7:10.  Given this information we can
start to generalize the type of ``commandments'' the author refers to when
using \entole.

And yet, other times the word ``commandment'' appears without any matching
mention of an actual commandment in the verse or in surrounding verses.
See Luke 1:6 for example.

\section{Methods}
36 of the total 67 occurences of \entole and its variants are
explicitly linked to a specific commandment or set of commandments, by
a specific commandment, or set of commandments, being referenced in or around
the verse in which \entole appears.

We take a conservative approach in determining when these explicit links
occur.  For example, the passage Matthew 22:34--40 is a conversation between
Jesus and a lawyer, in which \entole appears three times: verses 36, 38, and 40,
with commandments specifically referenced in verses 38 and 40.  Since the
speaker in verse 36 is not the same speaker as in verses 38 and 40, we count
only verses 38 and 40 as having explicit links to actual commandments.

We classify each of these 36 occurences according to the type of commandment,
or commandments,
to which each occurence belongs: ``Torah'', ``Decalogue'' (itself a part
of the Torah), and ``Other''.  The ``Other'' type includes man-made laws
and instructions from one person to another, among others.

\footnotesize
\renewcommand\arraystretch{1}
\begin{longtable}{lrr}
    \toprule
    Book & Chapter & Verse \\
    \midrule
    Matthew & 5 & 19 \\
    &15&3\\
    &19&17\\
    &22&36\\
    & &38\\
    & &40\\
    Mark&7&7\\
    & &8\\
    & &9\\
    &10&5\\
    & &19\\
    &12&28\\
    & &31\\
    Luke&1&6\\
    &15&29\\
    &18&20\\
    &23&56\\
    John&10&18\\
    &11&57\\
    &12&49\\
    & &50\\
    &13&34\\
    & &15\\
    & &21\\
    &15&10\\
    & &10\\
    & &12\\
    Acts&17&15\\
    Romans&7&8\\
    & &9\\
    & &10\\
    & &11\\
    & &12\\
    & &13\\
    &13&9\\
    1 Corinthians&7&19\\
    &14&37\\
    Ephesians&2&15\\
    &6&2\\
    Collosians&4&10\\
    1 Timothy&6&14\\
    Titus&1&14\\
    Hebrews&7&5\\
    & &16\\
    & &18\\
    &9&19\\
    2 Peter&2&21\\
    &3&2\\
    1 John&2&3\\
    & &4\\
    & &7\\
    & &7\\
    & &7\\
    & &8\\
    &3&22\\
    & &23\\
    & &23\\
    & &24\\
    &4&21\\
    &5&2\\
    & &3\\
    & &3\\
    2 John&1&4\\
    & &5\\
    & &6\\
    & &6\\
    Revelations&12&17\\
    &14&12\\
    \bottomrule
    \caption{Occurrences of \entole}
    \label{occurences}
\end{longtable}

\begin{table}
    \centering
    \footnotesize
    \renewcommand\arraystretch{1}
    \begin{tabularx}{\textwidth+40pt}{@{}lrrlX@{}}
    \toprule
    Book & Chapter & Verse & Type & Reason \\
    \midrule
    Matthew & 5&19 & Torah & Jesus uses the commandment as belonging to the Law \\
    &15&3 & Torah & Refers to Ex.\ 20:12, Ex.\ 21:17, Lev.\ 20:9 \\
    &19&17 & Torah & Five Decalogue commandments mentioned, plus Lev.\ 19:18 \\
    &22&26 & Torah & See Jesus's response in vv.\ 38--40\\
    & &38 & Torah & Lev.\ 19:18, Deut.\ 6:5\\
    & &40 & Torah & Lev.\ 19:18, Deut.\ 6:5\\
    Mark&7&7 & Other & phrase is ``commandments of men'' \\
    & &8 & Torah & Verse 10 refers to Ex.\ 20:12, Ex.\ 21:17, Lev.\ 20:9\\
    & &9 & Torah & Verse 10 refers to Ex.\ 20:12, Ex. 21:17, Lev.\ 20:9\\
    &10&5 & Torah & Refers to divorce law, Deut.\ 24:1 for example\\
    &12&28 & Torah & See Jesus's response in v.\ 31 \\
    & &31 & Torah & In His response, Jesus refers to Lev..\ 19:18, Deut.\ 6:5\\
    Luke&15&29 & Other & Refers to a command given from father to son\\
    &18&20 & Decalogue & Jesus refers to part of the Decalogue\\
    &23&56 & Decalogue & Refers to Sabbath keeping\\
    John&10&18 & Other & Jesus refers to a commandment given to Him from God the Father\\
    &11&57 & Other & Refers to command created by priests and Pharisees\\
    &12&49 & Other & Jesus refers to a commandment given to Him from God the Father\\
    &13&30 & Other & Jesus refers to a commandment given to Him from God the Father\\
    & &34 & Other & Jesus states that this is new commandment given by Him\\
    &15&12 & Other & Jesus states that this is His commandment\\
    Acts&17&15 & Other & Command from Silas and Timothy\\
    Romans&7&8 & Decalogue & Refers to coveting, a member of the Decalogue\\
    & &9 & Decalogue & Refers to coveting, a member of the Decalogue\\
    & &10 & Decalogue & Refers to coveting, a member of the Decalogue\\
    & &11 & Decalogue & Refers to coveting, a member of the Decalogue\\
    & &12 & Decalogue & Refers to coveting, a member of the Decalogue\\
    & &13 & Decalogue & Refers to coveting, a member of the Decalogue\\
    &13&9 & Decalogue & Word is used in reference to parts of the Decalogue.  ``Love thy neighbour'' is not called a commandment here.\\
    1 Corinthains&14&37 & Other & Paul refers to his own writings, in this context, as the commands\\
    Ephesians &6&2 & Decalogue & Honor your father and mother\\
    Colossians&4&10 & Other & These are instructions to Mark when he visits\\
    1 Timothy &6&14 & Other & Refers to the author's instructions to Timothy, given in previous verses\\
    Titus&1&14 & Other & Usage is ``commandments of men''\\
    Hebrews&7&5 & Torah & Refers to tithe\\
    & &16 & Torah & Refers to laws defining the Levitical priesthood\\
    & &18 & Torah & Refers to laws defining the Levitical priesthood\\
    &9&19 & Torah & May refer to Ex. 24:8.  Certainly not to the Decalogue\\
    1 John&3&22 & Other & Plural, referring to the commandment below in v.\ 23, but includes more\\
    & &23 & Other & The command here is to believe in the name of Jesus, and to love one another\\
    & &24 & Other & Plural, referring to the commandment above in v.\ 23, but includes more\\
    &4&21 & Other & The command here is that anyone who loves God must also love their brother and sister\\
    2 John&1&5 & Other & The command here is to love one another\\
    \bottomrule
\end{tabularx}
    \caption{Classification of \entole}
    \label{known}
\end{table}

\begin{table}
    \centering
    \footnotesize
    \renewcommand\arraystretch{1}
    \begin{tabular}{lrr}
    \toprule
    Book & Chapter & Verse \\
    \midrule
    Mark & 10 & 19 \\
    Luke & 1 & 6 \\
    John & 14 & 15 \\
    & & 21 \\
    & 15 & 10 \\
    1 Corinthians & 7 & 19 \\
    Ephesians & 2 & 15 \\
    2 Peter & 2 & 21 \\
    & 3 & 2 \\
    1 John & 2 & 3\\
    & & 4\\
    1 John & 5 & 2\\
    & & 3\\
    \bottomrule
\end{tabular}
    \caption{Unknown usages of \entole}
    \label{unknown}
\end{table}

We use the set of classified data as the basis for the data set used to train a neural network to classify the remaining 34
occurences of \entole and its variants.  
We use a multilayer perceptron, implemented by TensorFlow's \texttt{tf.estimator.DNNClassifier}.

The ouput layer consists of three neurons, one for each type.

The features, or inputs, categorise the metadata, or context, for each verse in which \entole occures. 
We include details about the location of the verse: the book, chapter, and verse; details about the author of the verse:
the traditional and inferred author of the book and the background of each (Jewish Christian, Gentile Christian, Christian), 
the intended audience of the book (Jewish CHristian, Gentile Christian, Christian); and details about dialogue, if any:
who is speaking, and who is listening.  This produces 10 neurons for the input layer.

\begin{table}
    \centering
    \renewcommand\arraystretch{1.3}
    \begin{tabular}{@{}lll@{}}
    \toprule
        Book & Traditional author & Traditional author's background \\
    \midrule
        Matthew & Matthew & Jewish Christian \\
        Mark & Mark & Jewish Christian \\
        Luke & Luke & Gentile Christian \\
        John & John & Jewish Christian \\
        Acts & Luke & Gentile Christian \\
        Romans & Paul & Jewish Christian \\
        1 Corintians & Paul & Jewish Christian \\
        Ephesians & Paul & Jewish Christian \\
        Collosians & Paul & Jewish Christian \\
        1 Timothy & Paul & Jewish Christian \\
        Titus & Paul & Jewish Christian \\
        Hebrews & Paul & Jewish Christian \\
        2 Peter & Peter & Jewish Christian \\
        1 John & John & Jewish Christian \\
        2 John & John & Jewish Christian \\
        Revelations & John & Jewish Christian \\
    \bottomrule
    \end{tabular}
    \caption{Traditional author and background}
    \label{traditionalauthor}
\end{table}

\begin{table}
    \centering
    \renewcommand\arraystretch{1.3}
    \begin{tabular}{@{}lll@{}}
    \toprule
        Book & Inferred author & Inferred author's background\\
    \midrule
        Matthew & Anon-Matthew & Jewish Christian~\cite{Duling2010},~\cite{Saldarini2003} \\
        Mark & Anon-Mark & Christian~\cite{Schroter2010} \\
        Luke & Anon-Luke & Gentile Christian \\
        John & Anon-John & Jewish Christian \\
        Acts & Anon-Luke & Gentile Christian \\
        Romans & Paul & Jewish Christian \\
        1 Corintians & Paul & Jewish Christian \\
        Ephesians & Pseudo-Paul-1 & Jewish Christian \\
        Colossians & Pseudo-Paul-2~\cite{Hooker2003} & Jewish Christian \\
        1 Timothy & Pseudo-Paul-3 & Jewish Christian \\
        Titus & Pseudo-Paul-3 & Jewish Christian \\
        Hebrews & Anon-Paul~\cite{Thiselton2003} & Jewish Christian \\
        2 Peter & Pseudo-Peter~\cite{McKnight2003} & Jewish Christian \\
        1 John & Anon-John-2~\cite{Painter2003} & Jewish Christian \\
        2 John & Anon-John-2~\cite{Painter2003} & Jewish Christian \\
        Revelations & John of Patmos & Jewish Christian~\cite{Stuckenbruck2003} \\
    \bottomrule
    \end{tabular}
    \caption{Inferred author and background}
    \label{author}
\end{table}

\begin{table}
    \centering
    \renewcommand\arraystretch{1.3}
    \begin{tabular}{@{}ll@{}}
    \toprule
        Book & Audience \\
    \midrule
        Matthew & Jewish Christian~\cite{Saldarini2003} \\
        Mark & Gentile Christian~\cite{Schroter2010} \\
        Luke & Gentile \& Jewish Christian~\cite{Thompson2010} \\
        John & Jewish Christian \\
        Acts & Gentile Christian \\
        Romans & Jewish Christian \\
        1 Corintians & Gentile Christian\tablefootnote{1 Corintians 12:2} \\
        Ephesians & Gentile Christian~\cite{Marshall2003} \\
        Colossians & Gentile Christian~\cite{Hooker2003} \\
        1 Timothy & Jewish Christian \\
        Titus & Gentile Christian~\cite{Perkins2003} \\
        Hebrews & Jewish Christian~\cite{Thiselton2003} \\
        2 Peter & Jewish Christian \\
        1 John & Gentile Christian~\cite{Painter2003} \\
        2 John & Gentile Christian~\cite{Painter2003} \\
        Revelations & Jewish Christian \\
    \bottomrule
    \end{tabular}
    \caption{Audience}
    \label{audience}
\end{table}

\begin{table}
    \centering
    \renewcommand\arraystretch{1.3}
    \begin{tabular}{@{}lrrll@{}}
    \toprule
        Book & Chapter & Verse & Speaker & Listener \\
    \midrule
        Matthew & 5 & 19 & Jesus & crowds \\
         & 15 & 3 & Jesus & scribes and Pharisees \\
         & 19 & 17 & Jesus & rich young man \\
         & 22 & 36 & lawyer & Jesus \\
         &  & 38 & Jesus & lawyer \\
         &  & 40 & Jesus & lawyer \\
        Mark & 7 & 7 & Jesus & scribes and Pharisees \\
         &  & 8 & Jesus & scribes and Pharisees \\
         &  & 9 & Jesus & scribes and Pharisees \\
         & 10 & 5 & Jesus & Pharisees \\
         & & 19 & Jesus & rich young man \\
         & 12 & 28 & scribe & Jesus \\
         & & 31 & Jesus & scribe \\
        Luke & 15 & 29& Jesus & scribes and Pharisees \\
        & 18 & 20 & Jesus & rich young man \\
        John & 10 & 18 & Jesus & Pharisees \\
         & 12 & 49 & Jesus & crowds \\
         & 13 & 30 & Jesus & crowds \\
         & & 34 & Jesus & disciples \\
         & 14 & 15 & Jesus & disciples \\
         & & 21 & Jesus & disciples \\
         & 15 & 10 & Jesus & disciples \\
         & & 12 & Jesus & disciples \\
    \bottomrule
    \end{tabular}
    \caption{Dialogue}
    \label{dialogue}
\end{table}

We use a single hidden layer.  There are multiple methods of determining how many neurons
to include in this layer.  In this paper we use two rules of thumb provided in~\cite{Heaton2008}.

\begin{eqnarray}
    N_h & = & \frac{2 * N_i}{3} + N_o\\
    N_h & = & \frac{N_i + N_o}{2}
\end{eqnarray}
where $N_h$ is the number of neurons in the hidden layer, $N_i = 10$ is the number of neurons in the input
layer, and $N_o = 3$ is the number of neurons in the output layer.  $N_h$ is equal to 10 and 7, respectively.

Run each test 10 times, take the averages and deviation.

\section{Results}
Here we can see that\ldots

\begin{table}
    \centering
    \renewcommand\arraystretch{1.3}
    \begin{tabular}{@{}lrrrrrrrr@{}}
    \toprule
        Book & Chapter & Verse & \multicolumn{2}{c}{Torah} & \multicolumn{2}{c}{Decalogue} & \multicolumn{2}{c}{Other} \\
        \cmidrule{4-5} \cmidrule{6-7} \cmidrule{8-9} & & & $\overline{x}$ & $\sigma^2$ & $\overline{x}$ & $\sigma^2$ & $\overline{x}$ & $\sigma^2$ \\
    \midrule
        Matthew & 22 & 36 & 0 & 0 & 0 & 0 & 0 & 0 \\
    \bottomrule
    \end{tabular}
    \caption{$N_h = 10$}
    \label{N_h10}
\end{table}

\begin{table}
    \centering
    \renewcommand\arraystretch{1.3}
    \begin{tabular}{@{}lrrrrrrrr@{}}
    \toprule
        Book & Chapter & Verse & \multicolumn{2}{c}{Torah} & \multicolumn{2}{c}{Decalogue} & \multicolumn{2}{c}{Other} \\
        \cmidrule{4-5} \cmidrule{6-7} \cmidrule{8-9} & & & $\overline{x}$ & $\sigma^2$ & $\overline{x}$ & $\sigma^2$ & $\overline{x}$ & $\sigma^2$ \\
    \midrule
        Matthew & 22 & 36 & 0 & 0 & 0 & 0 & 0 & 0 \\
    \bottomrule
    \end{tabular}
    \caption{$N_h = 7$}
    \label{N_h7}
\end{table}

\section{Discussion}
Test

\printbibliography
\end{document}
